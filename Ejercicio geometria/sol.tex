\documentclass[12pt,letterpaper,draft,oneside,onecolumn,titlepage]{article}
\usepackage[latin1]{inputenc}
\usepackage{amsmath}
\usepackage{amsfonts}
\usepackage{amssymb}
\author{Jose Ramirez}
\title{Problemas}
\begin{document}
	\begin{quotation}
		Let ABC be a triangle with $AB=AC$, and let $I$  be its incentre. The following
		holds: $BC=AB+AI$. Find all the possible values of $\angle BAC$.	
	\end{quotation}
	\begin{quotation}
		Using the usual notation, by the Briggs formula, we have:
		$$\sin \angle \frac{BAC}{2}=\sqrt{\frac{(p-b)(p-c)}{bc}} $$
		$$\Rightarrow \sin \angle \frac{BAC}{2}=\frac{(p-c)}{c}.$$

		hence $$AI=\frac{r}{\sin \angle \frac{BAC}{2}} \Rightarrow r=\frac{(a-c)(p-c)}{c}.$$

		Also, $$ r^2 = AI^2 - (p-a)^2 = (a-c)^2 - (p-a)^2 = \frac{3a^2}{4} - ac.$$

		This implies 	$$\left(\frac{(a-c)(p-c)}{c}\right)^2 = r^2 = \frac{3a^2}{4} - ac$$ 
						$$\Rightarrow 2(ac)^2 - 4ac^3 + 2ca^3 - a^4 = 0$$
						$$ \Rightarrow a(a-2c)(a^2-2c^2) = 0.$$
		Since $a\not=0$ and $a<2c$, we must have $a^2=2c^2$, therefore $\angle BAC=90$.
	\end{quotation}
\end{document}