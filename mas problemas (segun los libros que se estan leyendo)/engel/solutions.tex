\documentclass[12pt]{minimal}
\begin{document}
Invariants:

1.  Let $S_i$ be the sum of all the remaining integers in the list after the ith
    step; so, $S_0$ is the initial sum of all of them: $S_0 = 2n(4n - 1)$. The
    process keeps the parity of $S$ invariant. To prove it, suppose we choose
    $a$ and $b$ at the jth step. So, $S_{j + 1} = S_j - 2b$, assuming we
    substitute $a$ and $b$ with $a - b$, otherwise it would be $S_{j + 1} = S_j
    - 2a$. This shows that the parity of $S$ remains invariant to the last
    number, so, the last one will be an even one, as $S_0$ is.

10. Let $f(x_1, ..., x_n) = \sum{(x_i^2 - x_{i + 1}^2)}$, $x_{n + 1} = x_1$. Suppose
    we choose $x_1, x_2, x_3, x_4$ at a step, and switch $x_2$ and $x_3$. So,
    before the step we had $f_{old} = f(x_1, x_2, x_3, ...)$, and now, after the
    exchange, we have $f_{new} = f(x_1, x_3, x_2, ..., x_n)$, so $f_{new} -
    f_{old} = 2(x_1 x_3 + x_2 x_4 x_1 x_2 + x_3 x_4) = 2(x_1 - x_4)(x_2 - x_3)
    < 0$ by hypothesis. So, each time we make a switch, $f$ decreases. But,
    since $f$ is non-negative, it must come to a minimum, after which the
    numbers will stop changing.

Extremal principle:

5.  Suppose we have a tetrahedron ABCD solving the problem. Suppose AB is the
    longest edge of ABCD. Then, all four angles ABC, ABD, BAC, BAD must be acute,
    contradicting our initial assumption.

6.  Choose the face $f$ with the maximum number of sides. Suppose it has m sides.
    Then, this face is adjacent to $m$ other faces, making a total of $m+1$
    faces with $s$ sides, $3 \leq s \leq m$. So, by the pigeonhole principle, we
    must have two faces with the same number of sides in this set.

7.  So, we'll go here by parts:

    a)  No one hits the one who's farthest from the rest, since everyone shoots
        their nearest target.
    c)  They can't cross. Suppose some pair of bullet paths do, like A hits B,
        C hits D, and AB and CD cross. Then ABCD is a convex queadrilateral, and
        it can be proven that A should choose either C or D instead of B.

8.  Suppose the board has $R$ rooks holding the conditions for the problem. Now, 
    choose a row $r$ with the minimum number of rooks. Suppose it has $t$ rooks.
    Then, each of the $n - t$ columns sharing an unoccupied square with $r$ has
    at least $n - t$ rooks, the rest have at least $t - 1$ of them. So,
    according to this we have $R \geq (n - t)^2 + (t - 1)t + t =(n - t)^2 + t^2
    \geq \frac{((n - t) + t)^2}{2} = \frac{n^2}{2}$. Equality holds if $n = 2t$.
\end{document}