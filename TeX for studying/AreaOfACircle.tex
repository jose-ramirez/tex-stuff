\documentclass[11pt]{article}
\usepackage{amsthm}
\usepackage{amssymb}
\usepackage{amsmath}

\newtheorem{prf}{Theorem}

\title{Proof of the Area of a Circle Formula $A = \pi r^2$}
\author{nr1337}
\begin{document}
\maketitle

\begin{prf}
The area of a circle with radius $r$ is $\pi r^2$.
\end{prf}

\noindent {\bf Proof:} The equation of a circle centered at the origin is

$$
x^2 + y^2 = r^2,
$$

\noindent where $r$ is the radius.  We  write $y$ in terms of the variable $x$ and the constant $r$:

$$
\frac{x^2}{r^2} + \frac{y^2}{r^2} = 1
$$
$$
\frac{y}{r} = \sqrt{1-\frac{x^2}{r^2}}
$$
$$
y= r\sqrt {1-\frac{x^2}{r^2}}
$$

By symmetry, the area of a circle centered at the origin is four times the area of the circle between $(0,0)$ and $(r, 0)$ above the $x$-axis.  We can integrate to find the area ($A$):

$$
A = 4r\int_0^r \sqrt {1-\frac{x^2}{r^2}}\, dx
$$

To evaluate the antiderivative of $\displaystyle\sqrt {1-\frac{x^2}{r^2}}$, we make the substitutions:

$$
x = r \sin \theta
$$
$$
\theta = \arcsin \frac{x}{r}
$$
$$
dx = r\cos \theta\, d\theta
$$

Thus, our integral becomes:

$$
A=4r\int_0^r \sqrt {1-\frac{x^2}{r^2}}\, dx = 4r\int_0^{\pi/2} r\sqrt{1-\sin^2 \theta} \cos \theta\, d\theta
$$

 We can use the trigonometric identity $1 - \sin^2 \theta = \cos^2 \theta$:

$$
A=4r\int_0^{\pi/2} r\sqrt{1-\sin^2 \theta} \cos \theta\, d\theta= 4r^2\int_0^{\pi/2} \cos^2 \theta\, d\theta
$$

We then apply $\cos^2 \theta = \frac{1}{2}(1 + \cos 2\theta)$:

\begin{eqnarray*}
4r^2\int_0^{\pi/2} \cos^2 \theta\, d\theta &=& 4r^2\int_0^{\pi/2}  \frac{1}{2}(1 + \cos 2\theta) \,d\theta\\ 
& = & {2r^2\theta}\Bigg{|}_0^{\pi/2} + 2r^2\int_0^{\pi/2} \cos 2\theta \,d\theta\\
							      & = & \pi r^2 + 2r^2(\sin2\theta)\Bigg{|}_0^{\pi/2}\\
							      & = & \pi r^2
\end{eqnarray*}

Thus, the area of a circle with radius $r$ is $\pi r^2$.\hfill$\blacksquare$


 \end{document}