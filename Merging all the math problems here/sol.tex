\documentclass{article}
\title{Some problems I've solved through time}
\usepackage{amssymb}
\usepackage{amsmath}
\author{Jose Ramirez}
\setlength{\parindent}{0pt}
\pdfpagewidth 8.5in
\pdfpageheight 11in

\begin{document}
\maketitle

\section{Problems}
\label{sec:Problems}

\begin{enumerate}

\item Let ABC be a triangle with $AB=AC$, and let $I$  be its incenter. The following holds: $BC=AB+AI$. Find all the possible values of $\angle BAC$.

\item Find, in closed form, the value of $$f(m) = \sum_{k} \binom{2n + 1}{2k}\binom{m + k}{2n}.$$

\item Suppose $x, y, z > 0$ and $x + y + z = 1$. Then $$\left(1 + \frac{1}{x}\right)\left(1 + \frac{1}{y}\right)\left(1 + \frac{1}{z}\right) \geq 64.$$

\item Prove that $$(ab + bc + ac)(a + b + c)^4 \leq 27(a^3 + b^3 + c^3)^{2}$$ for $a, b, c \geq 0.$

\item Show that there are infinitely many positive integer numbers n such that $n^2 + 1$ has two positive divisors whose difference is $n$.

\item Let $x_{n + 1} = 4x_{n} - x_{n - 1}, x_0 = 0, x_1 = 1$, and $y_{n + 1} = 4y_{n} - y_{n - 1}, y_0 = 1, y_1 = 2$. Prove that $\forall n$, $y_{n}^2 - 3x_{n}^2 = 1$.

\item Let ABC be an acute triangle. Let $\Gamma$ be a semicircle with its diameter lying in $BC$, and tangent to both $AB$ and $AC$ at $D$ en $E$ respectively. Let $F$ and $G$ the intersections of $\Gamma$ with $BC$, $F$ being closer to $B$. Also, let $H$ the intersection of the segments $EF$ and $DG$. Then, the altitude from $A$ passes through $H$.

\item Start with the positive integers $1, \cdots, 4n - 1$. In one move you may replace any two integers by their difference. Prove that an even integer will be left after $4n - 2$ steps.

\item Consider the series expansion

$$\frac{1}{1 - 2x -x^2} = \sum_{n = 0}^{\infty}a_n x^n.$$

Prove that, for each integer $n \geq 0$, there is an integer $m$ such that

$$a_n^2 + a_{n + 1}^2 = a_m.$$

\item Find all positive integers $x, y$ such that
$$
	x^2 + y^2 = 3^x.
$$

\item If a cevian AQ of an equilateral triangle ABC is extended to meet the circumcircle at P, then $$\frac{1}{PQ} = \frac{1}{PB} + \frac{1}{PC}.$$

\end{enumerate}

\section{Solutions}
\label{sec:Solutions}

\begin{enumerate}

\item Using the usual notation, by the Briggs formula, we have: 
$$
	\sin \angle \frac{BAC}{2}=\sqrt{\frac{(p-b)(p-c)}{bc}}
	\Rightarrow \sin \angle \frac{BAC}{2}=\frac{(p-c)}{c},
$$
hence
$$
	AI=\frac{r}{\sin \angle \frac{BAC}{2}}
	\Rightarrow r=\frac{(a-c)(p-c)}{c}.
$$
Also,
$$
	r^2 = AI^2 - (p-a)^2 = (a-c)^2 - (p-a)^2 = \frac{3a^2}{4} - ac.
$$
This implies
\begin{eqnarray*}
\left(\frac{(a-c)(p-c)}{c}\right)^2 = r^2 = \frac{3a^2}{4} - a \\
\Rightarrow 2(ac)^2 - 4ac^3 + 2ca^3 - a^4 = 0 \\
\Rightarrow a(a-2c)(a^2-2c^2) = 0. \\
\end{eqnarray*}

Since $a\not=0$ and $a<2c$, we must have $a^2=2c^2$,
therefore $\angle BAC=90$.

\item Let
$$
	F(x) = \sum_{m} f(m)x^m.
$$

Summing over $m$, we get:

\begin{eqnarray*}
F(x) = \sum_{m}f(m) x^{m} = \sum_{m}\left(\sum_{k} \binom{2n + 1}{2k}\binom{m + k}{2n}\right) x^{m} = \\
\sum_{k} \binom{2n + 1}{2k} \left(\sum_{m}\binom{m + k}{2n} x^{m}\right) = \\
\frac{x^{2n}}{(1 - x)^{2n + 1}} \sum_{k}\binom{2n + 1}{2k}x^{-k}.
\end{eqnarray*}

Now our problem reduces to finding $[x^{m}]F(x)$; for that,
let $x = y^{2}$, so $[x^{m}]F(x) = [y^{2m}]F(y)$. Let's find $F(y^2)$:

\begin{eqnarray*}
F(y^2) = \frac{y^{4n}}{(1 - y^2)^{2n + 1}} \sum_{k} \binom{2n + 1}{2k}y^{-2k} = \\
\frac{y^{4n}}{2(1 - y^2)^{2n + 1}}\left( \left(1 + \frac{1}{y}\right)^{2n + 1} + \left(1 - \frac{1}{y}\right)^{2n + 1} \right) = \\
\frac{y^{4n}}{2(1 - y^2)^{2n + 1}}\left(\frac{(1 + y)^{2n + 1}}{y^{2n + 1}} - \frac{(1 - y)^{2n + 1})}{y^{2n + 1}} \right) = \\
\frac{1}{2y}\left(\frac{y^{2n}}{(1 - y)^{2n + 1}} - \frac{y^{2n}}{(1 + y)^{2n + 1}}\right).
\end{eqnarray*}

So,
$
	f(m) = [x^{m}]F(x) = [y^{2m}]F(y^2)
	= \frac{1}{2}\left( \binom{2m + 1}{2n} - (-1)^{2m + 1}\binom{2m + 1}{2n} \right)
	= \binom{2m + 1}{2n}.
$

\item So, by the power mean inequality we have
$$
	\left(1 + \frac{1}{x}\right)\left(1 + \frac{1}{y}\right)\left(1 + \frac{1}{z}\right)
	\geq \left(\frac{3}{\frac{x}{1 + x} + \frac{y}{1 + y} + \frac{z}{1 + z}}\right)^{3}
$$

and, since the function $f(x) = \frac{x}{1 + x}$ is concave for $x$ a positive
real number, by Jensen's inequality we have:

\begin{eqnarray*}
\frac{1}{3}\left(\frac{x}{1 + x} + \frac{y}{1 + y} + \frac{z}{1 + z}\right) \leq \frac{\frac{x + y + z}{3}}{1 + \frac{x + y + z}{3}} = \frac{1}{4} \\
\Rightarrow \left(\frac{x}{1 + x} + \frac{y}{1 + y} + \frac{z}{1 + z}\right) \leq \frac{3}{4} \\
\end{eqnarray*}

Therefore,
$$
	\left(1 + \frac{1}{x}\right)\left(1 + \frac{1}{y}\right)\left(1 + \frac{1}{z}\right) \geq
	\left(\frac{3}{\frac{3}{4}}\right)^{3} = 64.
$$

\item Transforming, the inequality can be rewritten as
\begin{eqnarray*}
27T[6, 0, 0] + 54T[3, 3, 0] \geq \\
\geq 2T[5, 1, 0] + 8T[4, 2, 0] + 9T[4, 1, 1] + \\
6T[3, 3, 0] + 44T[3, 2, 1] + 12T[2, 2, 2], \\
\end{eqnarray*}

which can be easily verified by repeatedly applying Muirhead's theorem.

\item
Let's suppose such $n$ exists; therefore there are integers
$a, b$ with $a < b$ such that $n^2 + 1 = ab$, and
$a - b = n$. We could say then that the following
equation should have an integer solution in $n$:
$$
	n^2 + 1 = a(a - n) \Rightarrow n^2 +an - (a^2 - 1) = 0,
$$
so for it to have integer solutions in $n$ its discriminant
must be a perfect square, i.e.:
$$
	a^2 + 4(a^2 - 1) = x^2 \Rightarrow x^2 - 5a^2 = -4.
$$
As a Pell equation with a minimal solution, $(1, 1)$, it
has infinitely many solutions; which are given by the
following formulas:
$$
	(x_{n + 1}, a_{n + 1}) = (9x_n + 20a_n, 4x_n + 9a_n).
$$	
So, for each $k \geq 0$, $a_k$ generates an $x_k$ that
generates an $n_k$ by the following formulae:
$$
	n_k = \frac{x_k - a_k}{2}, b_k = a_k - n_k.
$$

\item
First, let's prove the following recurrences:
$x_{n + 1} = 2x_n + y_n$, and $y_{n + 1} = 3x_n + 2y_n$,
both $\forall n \geq 0$.

The first one is pretty straightforward through induction.
The base cases $n = 0$ and $n = 1$ hold, so we can suppose
the equality holds for every $k \leq n$. Using this, let's
prove it for $n + 1$:

$
    y_{n + 1} = 4y_n - y_{n - 1} = 
    4(x_{n + 1} - 2x_n) - (x_n - 2x_{n - 1}) =
    (4x_{n + 1} - x_n) - (8x_n - 2x_{n - 1}) =
    (4x_{n + 1} - x_n) - 2(4x_n - x_{n - 1}) =
    x_{n + 2} - 2x_{n + 1},
$
so $x_{n + 2} = 2x_{n + 1} + y_{n + 1}$, as we claimed
initially.

For the second one, we'll use induction on $n$ again. The
base cases also hold for $n = 0, n = 1$ in this case, and
so we have:
$
    y_{n + 2} = 4y_{n + 1} - y_n = 
    2y_{n + 1} + (2y_{n + 1} - y_n) =
    2y_{n + 1} + (2(3x_n + 2y_n) - y_n) =
    2y_{n + 1} + (6x_n + 4y_n - y_n) =
    2y_{n + 1} + 3(2x_n + y_n) =
    2y_{n + 1} + 3x_{n + 1},
$
hence our recurrences are proved.

Now, to prove the statement, we see by using the above
recurrencies that, $\forall n \geq 1$:
$
    y_n^2 - 3x_n^2 =
    (3x_{n - 1} + 2y_{n - 1})^2 - 3(2x_{n - 1} + y_{n - 1})^2 =
    (4 - 3)y_{n - 1}^2 + 12 x_{n - 1} y_{n - 1} -
    12 x_{n - 1} y_{n - 1} + (9 - 12)x_{n - 1}^2 =
    y_{n - 1}^2 - 3x_{n - 1}^2,
$
so we can say that
$
    y_n^2 - 3x_n^2 = y_{n - 1}^2 - 3x_{n - 1}^2 =
    \cdots = y_1^2 - 3x_1^2 = y_0^2 - 3x_0^2 = 1,
$
as claimed.

\item
Let the altitude from $A$ intersect $BC$ at $J$. So, suppose $AJ$ does not
pass through $H$; then it must intersect each segment at different points;
like $EF$ at $H'$ and $DG$ at $H''$. So, at first we have the following
chain of equalities:


$\angle AEH' = \angle AEF = \angle EGF = \angle AH'E$ (since $\angle FEG =
\angle AJG = 90^{\circ}$, so EH'JG is cyclic), hence $AE = AH'$. Similarly, we
can conclude that $AD = AH''$, therefore (since $AD = AE$ by construction)
$AH' = AE = AD = AH''$, which in turn implies, since $H'$ and $H''$ lie on
the same line, that $H' = H''$, so $AJ$ passes through $H$ as claimed.

\item
Let $S_i$ be the sum of all the remaining integers in the list after the ith
step; so, $S_0$ is the initial sum of all of them: $S_0 = 2n(4n - 1)$. The
process keeps the parity of $S$ invariant. To prove it, suppose we choose
$a$ and $b$ at the jth step. So, $S_{j + 1} = S_j - 2b$, assuming we
substitute $a$ and $b$ with $a - b$, otherwise it would be $S_{j + 1} = S_j
- 2a$. This shows that the parity of $S$ remains invariant to the last
number, so, the last one will be an even one, as $S_0$ is.

\item So, from the series definition, we have

\begin{eqnarray*}
1 = (1 - 2x -x^2)\sum_{n = 0}^{\infty}a_n x^n = \\
\sum_{n = 0}^{\infty}a_n x^n - \sum_{n = 0}^{\infty}2a_n x^{n + 1} - \sum_{n = 0}^{\infty}a_n x^{n + 2} = \\
a_0 + (a_1 - 2a_0)x + \sum_{n = 2}^{\infty}(a_n -2a_{n - 1} - a_{n - 2}) x^n,
\end{eqnarray*}

so, comparing coefficients from both sides of this equation, we deduce that our
sequence $\{a_n\}$ is defined by $a_0 = 1, a_1 = 2, a_n = 2a_{n - 1} + a_{n - 2}$
for all $n \geq 2$.

By inspection (using python), we claim that $a_n^2 + a_{n + 1}^2 = a_{2n + 2}.$

Using standard theory, we know that $a_n = c_1r_1^n + c_2r_2^n$, where
$r_1, r_2$ are the roots of $x^2 - 2x - 1 = 0$, and $c_1, c_2$ depend on
$a_0, a_1, r_1, r_2$. I'll spare you the details of getting those numbers and
formatting/verifying the closed formula for $a_n$, and just state it:
$$
	a_n = \frac{1}{2\sqrt2}\bigl((1 + \sqrt{2})^{n + 1} - (1 - \sqrt{2})^{n + 1}\bigr).
$$

We'll use this formula to prove our claim.

So, let's start:
\begin{eqnarray*}
a_n^2 + a_{n + 1}^2 = \\
\bigl(\frac{1}{2\sqrt2}\bigl((1 + \sqrt{2})^{n + 1} - (1 - \sqrt{2})^{n + 1}\bigr)\bigr)^2 + \\
\bigl(\frac{1}{2\sqrt2}\bigl((1 + \sqrt{2})^{n + 2} - (1 - \sqrt{2})^{n + 2}\bigr)\bigr)^2 = \\
\frac{1}{8}\bigl( (1 + \sqrt{2})^{2n + 2} + (1 - \sqrt{2})^{2n + 2} + (1 + \sqrt{2})^{2n + 4} + (1 - \sqrt{2})^{2n + 4}\bigr) = \\ 
\frac{1}{8}\bigl((1 + \sqrt{2})^{2n + 2} + (3 + 2\sqrt{2})(1 + \sqrt{2})^{2n + 2} + \\
(1 - \sqrt{2})^{2n + 2} + (3 - 2\sqrt{2})(1 + \sqrt{2})^{2n + 2}\bigr) = \\
\frac{1}{8}\bigl((4 + 2\sqrt{2})(1 + \sqrt{2})^{2n + 2} + (4 - 2\sqrt{2})(1 - \sqrt{2})^{2n + 2}\bigr) = \\
\frac{1}{2\sqrt2}\bigl((1 + \sqrt{2})(1 + \sqrt{2})^{2n + 2} - (1 - \sqrt{2})(1 - \sqrt{2})^{2n + 2}\bigr) = \\
\frac{1}{2\sqrt2}\bigl((1 + \sqrt{2})^{2n + 3} - (1 - \sqrt{2})^{2n + 3}\bigr) = a_{2n + 2}.
\end{eqnarray*}

\item Let's try to find a more convenient expression for $x$ and $y$.

The given equation can be rewritten as
$$
	3^x = y^2 - x^2 = (y - x)(y + x),
$$
so we must have that $ y - x = 3^a, y + x = 3^b$, for some $a, b$ with
$a + b = x.$

So, solving for $x$ and $y$ in this case yields
$x = \frac{3^a(3^{b - a} - 1)}{2}, y = \frac{3^a(3^{b - a} + 1)}{2}.$

Now, from the expressions for both $x$ and $y$ (i.e., seeing that
$\frac{(3^{b - a} - 1)}{2}$ and $\frac{(3^{b - a} + 1)}{2}$ are consecutive
integers) we must have that $x = 3^ak,y = 3^a(k + 1)$ for some $k \geq 0$, and
with this, the initial equation can be rewritten as
$$
	3^{3^ak} = 3^{2a}(2k + 1).
$$

Now, we claim that the last equation can only have solutons if $k = 1.$
For this, an apparently well known lemma tells us that $3^a \geq 2a + 1$ with
equality iff $a \in \{ 0, 1 \}$.

So, if $k > 1$, by applying the lemma, then we have the following chain
of inequalities:
$$
	3^{3^ak} \geq 3^{(2a + 1)k} = 3^{2ak + k} > 3^{2a} 3^k > 3^{2a}(2k + 1),
$$
and so, we must have $k = 1.$ 

Now, $k = 1$ leaves us with $3^{3^a} = 3^{2a + 1}$, or $3^a = 2a + 1$, resulting
in $a = \{ 0, 1 \}$ (by the lemma).

Finally, using the values we found for $a$ and $k$, we get the following solutions:

\begin{eqnarray*}
   (k, a) = (1, 0) \Rightarrow (x, y) = (1, 2),	\\
   (k, a) = (1, 1) \Rightarrow (x, y) = (3, 6). 	\\
\end{eqnarray*}

It can be easily verified that these are indeed solutions to the original equation.

\item Let $AB = AC = BC = l.$

Since $\angle APB = \angle ACB$, triangles $BQP, AQC$ are similar; then $$\frac{PQ}{PB} = \frac{CQ}{l}.$$

Since $\angle APC = \angle ABC$, triangles $PQC, BQA$ are similar; then $$\frac{PQ}{PC} = \frac{BQ}{l}.$$

Summing, we get $$\frac{PQ}{PB} + \frac{PQ}{PC} = \frac{CQ}{l} + \frac{BQ}{l} = \frac{CQ + QB}{l} = \frac{l}{l} = 1.$$

Hence, $$PQ\left(\frac{1}{PB} + \frac{1}{PC}\right) = 1 \Rightarrow \frac{1}{PQ} = \frac{1}{PB} + \frac{1}{PC}.$$

\end{enumerate}

\end{document}