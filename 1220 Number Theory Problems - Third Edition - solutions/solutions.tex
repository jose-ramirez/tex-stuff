\documentclass[12pt]{minimal}
\usepackage{amsmath}
\begin{document}
	Problem 9.

	We are asked to show that there are infnitely many
	positive integer numbers n such that $n^2 + 1$ has two
	positive divisors whose difference is $n$.

	Now, suppose such $n$ exists; therefore there exist
	$a, b$ with $a < b$ such that $n^2 + 1 = ab$, and
	$a - b = n$. We could say then that the following
	equation should have an integer solution in $n$:
	$$
		n^2 + 1 = a(a - n) \Rightarrow n^2 +an - (a^2 - 1) = 0,
	$$
	so for it to have integer solutions in $n$ its discriminant
	must be a perfect square, i.e.:
	$$
		a^2 + 4(a^2 - 1) = x^2 \Rightarrow x^2 - 5a^2 = -4.
	$$
	As a Pell equation with a minimal solution, $(1, 1)$, it
	has infinitely many solutions; which are given by the
	following formulas:
	$$
		(x_{n + 1}, a_{n + 1}) = (9x_n + 20a_n, 4x_n + 9a_n).
	$$	
	So, for each $k \geq 0$ $a_k$ generates an $x_k$ that
	generates an $n_k$ by the following formulae:
	$$
		n_k = \frac{x_k - a_k}{2}, b_k = a_k - n_k.
	$$

	Problem 11.

	Now, we are asked to find all values of $x$ for which
	$f(x) = x^8 + 2^{2^x + 2}$ is prime. $x = 0$ gives
	$f(0) = 8$, and $x = 1$ gives $f(1) = 17$, so we'll assume
	$x > 1$ in order to show that $17$ is the only possible
	prime value of $f(x)$.

	To show this, note that the given factorization holds
	under our initial assumption:
	$$ x^8 + 2^{2^x + 2} =
	(x^4 + 2^{2^{x - 1} + 1} - x^2 2^{2^{x - 2} + 1})
	(x^4 + 2^{2^{x - 1} + 1} + x^2 2^{2^{x - 2} + 1})$$
	(this follows from Sophie Germain's identity).

	Now, to be sure that $f(x)$ isn't prime, it suffices to
	note that
	$
		x^4 + 2^{2^{x - 1} + 1} - x^2 2^{2^{x - 2} + 1} =
		(x^2 - 2^{2^{x - 2}})^2 + 2^{2^{x - 1}} > 1
	$, so both factors above are greater than $1$ for $x > 1$;
	therefore $17$ is the only possible prime of the form
	$f(x)$.

	Problem 15.

	So, from the original statement we have the following
	sequences:
	$x_{n + 1} = 4x_{n} - x_{n - 1}, x_0 = 0, x_1 = 1$, and
	$y_{n + 1} = 4y_{n} - y_{n - 1}, y_0 = 1, y_1 = 2$. We
	are asked to prove that
	$\forall n$, $y_{n}^2 - 3x_{n}^2 = 1$.

	So, in order to prove that, we're just gonna prove the
	following recurrences:
	$x_{n + 1} = 2x_n + y_n$, and $y_{n + 1} = 3x_n + 2y_n$,
	both $\forall n \geq 0$.

	The first one is pretty straightforward through induction.
	The base cases $n = 0$ and $n = 1$ hold, so we can suppose
	the equality holds for every $k \leq n$. Using this, let's
	prove it for $n + 1$:

	$
		y_{n + 1} = 4y_n - y_{n - 1} = 
		4(x_{n + 1} - 2x_n) - (x_n - 2x_{n - 1}) =
		(4x_{n + 1} - x_n) - (8x_n - 2x_{n - 1}) =
		(4x_{n + 1} - x_n) - 2(4x_n - x_{n - 1}) =
		x_{n + 2} - 2x_{n + 1},
	$
	so $x_{n + 2} = 2x_{n + 1} + y_{n + 1}$, as we claimed
	initially.

	For the second one, we'll use induction on $n$ again. The
	base cases also hold for $n = 0, n = 1$ in this case, and
	so we have:
	$
		y_{n + 2} = 4y_{n + 1} - y_n = 
	    2y_{n + 1} + (2y_{n + 1} - y_n) =
		2y_{n + 1} + (2(3x_n + 2y_n) - y_n) =
		2y_{n + 1} + (6x_n + 4y_n - y_n) =
		2y_{n + 1} + 3(2x_n + y_n) =
		2y_{n + 1} + 3x_{n + 1},
	$
	hence our recurrences are proved.

	Now, to prove the statement, we'll use the above
	recurrences, and we find this, $\forall n \geq 0$:
	$
		y_n^2 - 3x_n^2 =
		(3x_{n - 1} + 2y_{n - 1})^2 - 3(2x_{n - 1} + y_{n - 1})^2 =
		(4 - 3)y_{n - 1}^2 + 12 x_{n - 1} y_{n - 1} -
		12 x_{n - 1} y_{n - 1} + (9 - 12)x_{n - 1}^2 =
		y_{n - 1}^2 - 3x_{n - 1}^2,
	$
	so we can say that
	$
		y_n^2 - 3x_n^2 = y_{n - 1}^2 - 3x_{n - 1}^2 =
		\cdots = y_1^2 - 3x_1^2 = y_0^2 - 3x_0^2 = 1,
	$
	as needed.

	Problem 71.

	Here we have three integers $m, n, d$ of which we
	know that $d \mid mn^2 + 1$ and $d \mid m^2n + 1$. We
	are asked to show that $d \mid m^3 + 1$ and
	$d \mid n^3 + 1$. For this, note that
	$d \mid m(mn^2 + 1) - n(m^2n + 1) = m - n$,
	so for one case we have:
	$
		d \mid m^2(m - n) \Rightarrow
		d \mid m^3 - m^2n \Rightarrow
		d \mid m^3 + 1 - (m^2n + 1) \Rightarrow
		d \mid m^3 + 1
	$
	, as needed. The other part is proved analogously.

	Problem 153.

	We are asked to prove that $9^{9^9}$ and $9^{9^{9^9}}$
	have the same last two digits.

	To do this, let's find $9^{9^9} \pmod {10^2}$.
	We know by Euler's theorem that
	$9^{\phi(10^2)} \equiv 1 \pmod {10^2}$, or
	$9^{40} \equiv 1 \pmod{10^2}$, and
	$9^2 \equiv 1 \pmod {40}$, therefore:
	$
		9^9 \equiv 9 \pmod{40} \Rightarrow
		9^{9^9} \equiv 9^9 \pmod{10^2}
	$.
	Now, to find $9^{9^{9^9}} \pmod{10^2}$, we start
	from $9^2 \equiv 1 \pmod{16}$ up, like this:
	$
		9^9 \equiv 9 \pmod{16} \Rightarrow
		9^{9^9} \equiv 9^9 \equiv 9 \pmod{40} \Rightarrow
		9^{9^{9^9}} \equiv 9^9 \pmod{10^2}
	$, and so both numbers must have the same last two
	digits.
\end{document}