\documentclass{article}
\author{Jose Ramirez}
\setlength{\parindent}{0pt}
\pdfpagewidth 8.5in
\pdfpageheight 11in
\usepackage{amssymb}
\usepackage{amsmath}
\usepackage{parskip}
\setlength{\parskip}{\baselineskip}
\begin{document}


$F_1, F_2$ and $F_3$ are force vectors for whichs the following equalities hold:
\begin{align*}
	F_1 + F_2 + F_3 = 0, \\
	F_1 \times r_1 + F_2 \times r_2 + F_3 \times r_3 = 0.
\end{align*}
Then $F_1, F_2$ and $F_3$ are coplanar, i.e., there's a plane containing these vectors.  


So, let's define $r_{ij} = r_i - r_j\ \forall i, j \in [1 \mathrm{ ... } 3], i < j $.

Let's prove some things first:

\begin{enumerate}
\item $r_{23} \times F_3 = r_{12} \times F_1$.
\begin{align*}
	r_{23} \times F_3 = r_2 \times F_3 - r_3 \times F_3 = \\
	r_2 \times F_3 + r_1 \times F_1 + r_2 \times F_2 = \\
	r_2 \times (F_2 + F_3) + r_1 \times F_1 = \\
	r_1 \times F_1 - r_2 \times F_1 = r_{12}\times F_1.
\end{align*}

\item $r_{23} - F_3, r_{12}, F_1$ are coplanar.
$$0 = (r_{23} - F_3) \cdot (r_{23} \times F_3) = (r_{23} - F_3) \cdot (r_{12} \times F_1),$$
so the aforementioned vectors lie on some plane; let's call it $\pi_1.$

\item $r_{23}, r_{12}, F_1$ are coplanar as well.
$$0 = r_{23} \cdot (r_{23} \times F_3) = r_{23} \cdot (r_{12} \times F_1),$$
so these vectors also lie on the same plane, let's call it $\pi_2.$
\end{enumerate}

By items 2 and 3, $\pi_1 = \pi_2 = \pi$, since $\pi_1$ and $\pi_2$ have both
$F_1$ and $r_{12}$ in common, which means that $r_{23}, (r_{23} - F_3)$ $\in \pi$,
so, $r_{23} - (r_{23} - F_3) = F_3 \in \pi$, so $-F_3 - F_1 = F_2 \in \pi$ as well.

So, $F_1, F_2, F_3 \in \pi$, as needed.

If $F_1 \parallel F_3$, then $ F_2 = -(F_1 + F_3)$ is such that $F_2 \parallel F_3$.

If not, then their supporting lines meet at a point, call it $P$. Choosing $P$ as
a new origin, we get that $r_1 \times F_1 = r_3 \times F_3 = 0$, which implies
$r_2 \times F_2 = 0$ as well.

This means that $F_2$ lies on the supporting line of $r_2$, call it $l_2$, therefore
the supporting lines of the force vectors concur at $P$.
\end{document}