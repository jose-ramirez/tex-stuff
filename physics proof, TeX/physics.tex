\documentclass[12pt]{minimal}
\usepackage{amsmath}
\begin{document}
	$F_1, F_2$ and $F_3$ are 3 force vectors such that $F_1 + F_2 + F_3 = 0$ and
	$F_1 \times r_1 + F_2 \times r_2 + F_3 \times r_3 = 0$.
	Then $F_1, F_2$ and $F_3$ are coplanar, i.e., there's a plane containing the
	these vectors.

	So, set $r_{ij} = r_i - r_j \forall i, j \in [1\mathrm{..}3], i < j $. There
	is a plane $\pi$ containing these $r_{ij}$. 

	So, we claim that $r_{23} \times F_3 = r_{12} \times F_1$.

	\begin{align*}
		r_{23} \times F_3 = r_2 \times F_3 - r_3 \times F_3 = \\
		r_2 \times F_3 + r_1 \times F_1 + r_2 \times F_2 = \\
		r_2 \times (F_2 + F_3) + r_1 \times F_1 = \\
		r_1 \times F_1 - r_2 \times F_1 = r_{12}\times F_1.
	\end{align*}

	So, we know the following:

	\begin{align*}
		0 = (r_{23} - F_3) \cdot (r_{23} \times F_3) = \\
		(r_{23} - F_3) \cdot (r_{12} \times F_1),
	\end{align*}

	so, $r_{23} - F_3, r_{12}, F_1$ belong to a plane, call it $\pi_1$, and

	\begin{align*}
		0 = r_{23} \cdot (r_{23} \times F_3) = \\
		r_{23} \cdot (r_{12} \times F_1),
	\end{align*}

	hence $r_{23}, r_{12}, F_1$ belong to a plane, call it $\pi_2$.

	So, $\pi_1 = \pi_2$ since they share $F_1, r_{12}$.

	so, $r_{23} - (r_{23} - F_3) = F_3$ belongs to this common plane, and
	hence  $-F_3 - F_1 = F_2$ as well.

	So all force vectors belong to the same plane, as needed.

	If $F_1 \parallel F_3$, then $ F_2 = -(F_1 + F_3)$ is such that $F_2 \parallel F_3$.

	If not, then their supporting lines meet at a point, call it $P$. Choosing $P$ as
	a new origin, we get that $r_1 \times F_1 = r_3 \times F_3 = 0$, which implies
	$r_2 \times F_2 = 0$ as well.

	This means that $F_2$ lies on the supporting line of $r_2$, call it $l_2$, therefore
	the supporting lines of the force vectors concur at $P$.
\end{document}